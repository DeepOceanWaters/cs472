\documentclass[letterpaper,10pt,titlepage]{article}

\usepackage{graphicx}                                        
\usepackage{amssymb}                                         
\usepackage{amsmath}                                         
\usepackage{amsthm}                                          

\usepackage{alltt}                                           
\usepackage{float}
\usepackage{color}
\usepackage{url}

\usepackage{balance}
\usepackage[TABBOTCAP, tight]{subfigure}
\usepackage{enumitem}
\usepackage{pstricks, pst-node}

\usepackage{geometry}
\geometry{textheight=8.5in, textwidth=6in}

%random comment

\newcommand{\cred}[1]{{\color{red}#1}}
\newcommand{\cblue}[1]{{\color{blue}#1}}

\usepackage{hyperref}
\usepackage{geometry}

\def\name{Colin Bradford, Bryce Holley}

%pull in the necessary preamble matter for pygments output
%\input{pygments.tex}

%% The following metadata will show up in the PDF properties
%\hypersetup{
%  colorlinks = true,
%  urlcolor = black,
%  pdfauthor = {\name},
%  pdfkeywords = {cs311 ``operating systems'' files filesystem I/O},
%  pdftitle = {CS 311 Project 1: UNIX File I/O},
%  pdfsubject = {CS 311 Project 1},
%  pdfpagemode = UseNone
%}

\begin{document}
\begin{description}
    \item[3.1 Why is the program counter a pointer and not a counter?]
    \item[3.2 Explain the function of the following registers in a CPU: PC, MAR, MBR, IR]
    \item[3.3 For each of the following 6-bit operations, calculate the values of the C, Z, V, and N flags.]
    \item[3.10]
    \item[3.17]
    \item[3.18]
    \item[3.19]
    \item[3.25]
    \item[3.39]
    \item[3.51]
\end{description}

%input the pygmentized output of mt19937ar.c, using a (hopefully) unique name
%this file only exists at compile time. Feel free to change that.
%\input{__mt19937ar.c.tex}
\end{document}